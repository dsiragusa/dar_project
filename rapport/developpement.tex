\section{Développement}
Après avoir testé tout le panel des IDEs compatibles JavaEE, nous nous sommes arrêté sur IntelliJ-Idea Ultimate a l'aide de licences étudiantes gratuites.
L'outil permet de créer des projets Java web, possède des plugins TomCat, Maven, Git. Il fonctionne sur toutes nos machines.
Une fois l'environnement de développement choisi, nous commençons d'abord par essayer de concevoir le site 'from scratch', sans aucun framework, avec les Servlets et JSP comme seules fondations.
Cette première esquisse nous permet de de commencer le développement du client et ainsi finaliser la conception de la base de donnée grâce des mock-ups simples.
Nous réalisons ensuite l'ampleur de la tâche qu'il nous reste a abattre et choississons de nous tourner vers Spring et Hibernate après une très brêve étude des non-alternatives autorisées par le sujet.

\subsection{Schéma de l'application}
Ici sera le schéma de l'application...

% ne pas seulement parler des choix technologiques
\subsection{Choix techniques}
Ici sera la description de nos choix techniques...

% courte description des tehcnologies, surtout justifier leur utilisation
\subsection{Technologies}
Ici sera la description des technologies utilisées...
