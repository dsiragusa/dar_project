\documentclass[a4paper,11pt]{article}
\usepackage[francais]{babel}
\usepackage{amssymb}
\usepackage{graphicx}
\usepackage{listings}
\usepackage[hidelinks]{hyperref}
\usepackage[utf8]{inputenc}
\usepackage[T1]{fontenc}

\usepackage{graphicx}
\usepackage{float}


\hypersetup{
	colorlinks = true,
	linkcolor = red,
	urlcolor = blue,
	filecolor = blue
}

\author{Boris Berger\\Daniele Siragusa\\Giorgi Shavgulidze\\Thomas Perez}
\title{Développement d'Application Réticulaire\\Rapport\\PariServices}

\begin{document}

\maketitle
\newpage

\tableofcontents
\newpage

\section{Introduction}
PariServices est une application web offrant à ses utilisateurs la possibilité de demander et/où de rendre des services sur Paris. L'application permet entre autre à ses utilisateurs de localiser via l'API Google Maps le lieux auquel le service devra être rendu. Elle offre par ailleurs tout un système de gestion des comptes utilisateurs, qui permet à chaque memebre de savoir ses services demandé et offerts en cours.
\newline
L'application a pour finalité de pallier à deux manques de la société actuelle que sont le manque de temps et le manque d'argent. L'idée est de permettre à la première catégorie de déléguer certaines de ses tâches, moyennant finance, et d'offrire à la seconde catégorie une source de revenu. L'application est donc fortement orientée communauté.

\section{Manuel utilisateur}
Ici sera le manuel utilisateur...
\subsection{Interface}
Ici sera la description de l'interface...
\subsection{Case d'utilisation}
Ici sera la description des cas d'utilisation...


\section{Développement}
Après avoir testé tout le panel des IDEs compatibles JavaEE, nous nous sommes arrêté sur IntelliJ-Idea Ultimate a l'aide de licences étudiantes gratuites.
L'outil permet de créer des projets Java web, possède des plugins TomCat, Maven, Git. Il fonctionne sur toutes nos machines.
Une fois l'environnement de développement choisi, nous commençons d'abord par essayer de concevoir le site 'from scratch', sans aucun framework, avec les Servlets et JSP comme seules fondations.
Cette première esquisse nous permet de de commencer le développement du client et ainsi finaliser la conception de la base de donnée grâce des mock-ups simples.
Nous réalisons ensuite l'ampleur de la tâche qu'il nous reste a abattre et choississons de nous tourner vers Spring et Hibernate après une très brêve étude des non-alternatives autorisées par le sujet.

\subsection{Schéma de l'application}

\includegraphics[width=\linewidth]{ApplicationSchema.jpg}

\subsection{Technologies}

\subsubsection{Spring}
Nous avons en particulier profité de Spring Security, qui propose des mesures de securité pour l'application appliquées de manière automatique: celles ci incluent la protection contre les attaques CSRF (en toujours ajoutant un champ contenant un token aux forms) et une gestion securisée des utilisateurs (système d'authentification, roles des comptes et cryptation secure des mots de passe à l'aide d'un hash sha-256 et d'un salt).

\subsubsection{Hibernate}
En ce qui concerne les données persitantes, nous avons utilisé l'ORM Hibernate en conjoction avec JPA. Le premier nous garanti un système automatique de mapping entre les objets et la BDD en écriture et en lecture; le deuxieme nous permet d'ecrire les configurations de persistance des entités grace a des annotations Java plutot que des fichiers configuration XML. Ce qui est beaucoup plus pratique.
Nous avons reussi à beaucoup modulariser le code en suivant l'architecture MVC plus quelques services et annotation (surtout pour ce qui concerne la validation des entités et l'injection d'entités dans les controleurs).

\subsubsection{Thymeleaf}
Nous profitons de ce moteur de templating pour générer dynamiquement les pages HTML en fonctions des valeurs des attributs mis à jour dans les Controller.

\subsubsection{MySQL}
Nous avons choisit MySQL a cause de multiples raisons. Il est gratuit, puisant, efficace et compatible avec Hibernate.
Mais ce ne sont pas seulement ces avantages qui ont motivé notre choix.

\includegraphics[width=\linewidth]{schemaBDD.png}
\begin{center}
\textit{Schéma de notre BDD.}
\end{center}

Le schéma de bases des données explique tout. Il n’est pas énormément complexe mais les relations entre les tables sont nombreuses. Ca veut dire qu'une base de données orientée documents n’est pas ni optimale ni facile à utiliser. Par exemple, l'entité Service a des liaison avec avec les autres tables. Dans une base de données orientée documents, les résultats des "query" pourraient être très grand et peu flexibles. De plus, certaines données pourraientt être dupliquées inutilement, comme « Category » et « Tag ».
\newline
Pour toutes ces raisons nous avons choisit MySQL.

\subsubsection{jQuery et Bootstrap}
Pour la partie client de l'application, nous avons choisit d'utiliser les frameworks jQuery et Bootstrap. Le premier nous permet de simplifier l'écriture de code java-script tout en s'assurant la compatibilité avec les differents navigateurs. L'autre nous donne accès à un ensemble de classes CSS qui rendent notre interface "responsive", en s'adaptant à toute taille d'écran (des smartphone aux plus grands écrans des ordinateurs de bureau).


\section{Fonctionnalités}

\subsection{Implémentées}

\subsubsection{Comptes utilisateurs}
Notre application possède un système complet de gestion des comptes utilisateurs. Les parties enregistrement et identification de l'utilisateur nous ont été fournies par le framework Spring. L'avantage est que ces fonctionnalités sont extrêmement sûres en terme de sécurité. Assurées via Spring Security, le mot de passe de l'utilisateur est chiffré via une fonction de hachage SHA-256 et un salt est également généré. L'identifaction de l'utilisateur se fait à l'aide de son adresse email et de son mot de passe.
\newline
Nous contrôlons alors si l'utilisateur est bien identifié lorsqu'il tente d'accéder aux fonctionnalités du site. De plus, après son inscription, un mail est envoyé à l'utilisateur sur l'adresse mail qu'il a renseigné afin de lui demander de la confirmer.

\subsubsection{Création d'une demande de service.}
Les créations de demande de service permettent à l'utilisateur de renseigner un titre et une description pour le service. Il doit également renseigner l'adresse  à laquelle il souhaite que le service lui soit rendu. Il associera également le service à une catégorie et lui attribuera des tags. Nous utilisons JQuery pour l'ajout de tags. Il doit également spécifier la date de limite de fin des candidatures et la date limite pour rendre le service.

\subsubsection{Rechercher un service à rendre}
Les recherches de services utilisent les chmps remplis par le demandeur lors de la création du service. L'utilisateur peut alors naviguer à travers les services disponibles en spécifiant un titre, et une catégorie. Le résultat de la recherche affiche notament la date limite pour les candidatures et le rendu du service. L'utilisateur peut également choisir le nombre d'annonce à afficher par page.

\subsection{Non implémentées}

\subsubsection{Candidater pour un service}
Cette fonctionnalité n'a pas été implémentée par manque de temps. L'idée était que le demandeur de service puisse sélectionner le volontaire de son choix parmis ceux s'étant proposés pour le service. Le volontaire sélectionné serait alors notifié de sa sélection via l'envoie d'un mail à son adresse.


\section{Compléments}

\subsection{Extensions possibles}

\subsubsection{Système de notifications}
Pour l'heure, les volontaires sont notifiés via l'envoie d'un mail lorsqu'ils sont sélectionnés pour une tâche. L'idée serait de se changer ce système pour des notifications internes au site. Ces notifications seraient signalées à l'utilisateur via un petit popup aparaissant dans le header de la page, près du lien vers le profil de l'utilisateur. Lors de l'arrivée sur son profil, l'utilisateur serait alors informé de son acceptation via un message plus explicite.

\subsubsection{Système d'administration}
On imagine bien qu'une application permettant aux utilisateurs de demander des services contre de l'argent pourrait facilement être utilisée à des fins illégales. Pour prévenir ce genre d'abus, plusieurs solutions nous semble possible:

\begin{itemize}
\item la mise en place d'administrateurs chargés de valider et de supprimer les annonces dites "illégales". Cela supposerait plusieurs ajouts, notamment une nouvelle table dans la base de donnée qui stockerait les administrateurs. De nouveaux boutons permettant de supprimer ou de modifier une annonce apparaîteraient alors lorsqu'un administrateur arriverait sur la page d'une annonce. Il faudrait également ajouter la possibilité de bannir un utilisateur, avec l'ajout d'un bouton accessibles uniquement aux administrateurs sur la page de profil d'un utilisateur. Une dernière idée serait d'ajouter sur la page des services un bouton "Signaler l'annonce". Visible par tous, ce boton permettrait d'envoyer une notification aux  administrateurs du site qui leur permettrait de venir contrôlé l'annonce.
\item l'implantation d'une tâche automatisée chargée de parcourire les annonces dus site à interval régulier et de supprimer automatiquement les annonces. La tâche pourrait par exemple parser les champ des annonces à la recherche de mots révélateurs. Cette solution poserait de nombreux problèmes de concurrence, notament avec notre tâche ettant à jour les annonces ayant dépassé leur date de rendu.
\end{itemize}

\subsubsection{Calcul du temps nécessaire à la satisfaction d'un service}
L'API Google Maps permet de connaître la distance séparant deux points sur la carte. Nous pourrions utiliser cette information qui, couplée au moyen de transport, pourrait nous permettre d'offrire à l'utilisateur une estimation du temps qu'il mettra à accomplir le service. Nous pourrions alors, en comparant à la date courante de l'utilisateur, signaler à l'utilisateur qu'il ne pourra pas accomplir ce service dans les temps.

\subsubsection{Intégration d'un système de paiement}
Actuellement, le paiement des services est laissé à la discrétion des utilisateurs. Nous pourrions intégrer la mécanique de paiement directement au site via un service comme paypal par exemple. Il faudrait alors notament se poser la question de la politique de paiement. Le volontaire pourrais par exemple être payer la moitié de la somme au moment ou il est sélectionner et l'autre moitié après confirmation du client que le service a bien été rendu.

\subsubsection{Intégration d'un système de réputation}
Permettre aux utilisateurs de s'évaluer entre eux au fil des transactions leur permettrait de mieux sélectionner leurs intermédiaires. On pourrait imaginer que le système de réputation soit diviser en deux: en tant que volontaire et en tant que demandeur. La réputation prendrait la forme d'un score global avec des évaluations au format texte.

\subsubsection{Respect de la norme ARIA}
De part son concept, notre application s'adresse tout particulièrement aux personnes ne pouvant pas sortir de chez elles du fait d'un handicap. Or, la norme ARIA offre un ensemble de propriétés qui, si elles sont respectées, garantissent l'accessibilité du site pour les personnes handicapées. Le non-respect de cette norme nous couperais donc l'accès à une énorme partie de notre publique potentiel. L'application de la norme n'est néanmoins pas immédiat et son respect impliquerais de nombreuses modifications dans l'interface de notre site.

\subsection{Monétisation}
La monétisation de notre application découle directement de notre amélioration intégrant le système de paiement à notre site. L'idée serait alors de prélever un pourcentage sur chaque transaction effectuée. Une autre idée serait d'offrire des services supplémentaires payant. Par exemple l'ajout de photos supplémentaires pour les annonces ou un système permettant aux utilisateurs de programmer automatiquement l'ajout d'annonce de façon périodique.


\section{Conclusion}
Notre application nous a permis de manipuler de nombreuses technologies que jusque là aucun membre du groupe n'avait utiliser. Nous avons pu constater la puissance des technologie Spring et Hbernate qui, combinées, offrent une élégance ainsi qu'un puissant contrôle pour le développement de l'application. Nous avons par ailleurs apprécier de découvrire ces technologies particulièrements populaire dans le monde de l'entreprise. Le groupe a également acquis une meilleure vision d'ensemble de c qu'est le développement d'une application web à l'échelle commerciale.

\end{document}
